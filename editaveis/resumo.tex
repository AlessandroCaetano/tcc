\begin{resumo}
 A utilização de metodologias ágeis para o desenvolvimento de software vem sendo
 uma crescente nos últimos anos, com isso, trazendo a tona as dificuldades intrínsecas 
 desta atividade que também estão presentes na metodologia ágil. Uma das
 dificuldades que se destacam e é uma constante em todos os processos, independente
 de metodologia, é a dificuldade de planejar entregas, ou releases. Para reduzir o
 impacto das incertezas dentro de projetos, principalmente os de grande complexidade
 e níveis de abstração, foram propostas pela literatura diversos métodos e tecnicas,
 entre elas a utilização de análise de dados para tomar decisões e planejar o futuro.
 Este trabalho propõe da utilização de \textit{Software Analytics}, para a anáĺise de
 relevancia de \textit{issues} de repositórios de código aberto com a utilização de
 algoritmos de \textit{Page Ranking} e centralidade de redes para apoiar a tomada
 de decisões de planejamento de sprint e a priorização de atividades.
 \vspace{\onelineskip}
    
 \noindent
 \textbf{Palavras-chaves}: Release Planning. Software Analytics. Page Ranking, Centralidade de Redes.
\end{resumo}
