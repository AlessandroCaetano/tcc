\chapter{Introdução}
\label{int}
\section{Contexto}
\label{int:con}
Os métodos de desenvolvimento ágeis surgiram como uma alternativa aos métodos de desenvolvimento chamados de tradicionais. Ao invés de planejar, analisar e projetar para um futuro em médio e longo prazo, a proposta dos métodos ágeis é que essas atividades, por exemplo, sejam realizadas continuamente em ciclos de produção \footnote{Neste trabalho os termos ciclo de produção, iteração e sprint são utilizados como sinômino} curtos, de 2 a 4 semanas, durante todo o ciclo de vida do projeto.
Para representar essa nova forma de olhar para o desenvolvimento de software, quatro valores foram propostos e enfatizados por um grupo de experientes desenvolvedores que se tornaram signatários do conhecido manifesto ágil \cite{beckManifesto}:
\begin{itemize}
    \item \textbf{Indivíduos e interação} entre eles mais que processos e ferramentas.
    \item \textbf{Software em funcionamento} mais que documentação abrangente.
    \item \textbf{Colaboração com o cliente} mais que negociação de contratos.
    \item \textbf{Responder a mudanças} mais que seguir um plano.
\end{itemize}

No ciclo de desenvolvimento ágil o cliente direciona as \textit{releases} ao priorizar os requisitos, representados pelas histórias de usuário, que possuem maior valor para o negócio. A cada iteração, o cliente escolhe um conjunto de  histórias que devem ser implementadas, respeitando a capacidade produtiva do time. A partir disso, são identificadas as atividades necessárias para que uma determinada história seja desenvolvida e estas são atribuídas aos programadores. São atividades de diferentes áreas da engenharia de software, como por exemplo: projetar, codificar, testar, medir, implantar, gerenciar configurações do software. Ao final da release, uma versão de produto de software é entregue ao cliente e o ciclo se inicia novamente.

O planejamento de release é uma atividade essencial realizada no início de cada novo ciclo de produção. Nesse momento são definidas as funcionalidades que serão implementadas para próxima versão de liberação do produto, além dos demais recursos do projeto que deverão ser alocados. Mesmo com o alto nível de interação entre as pessoas, característico em projetos de desenvolvimento ágil, o planejamento de release é uma tarefa difícil tanto computacionalmente quanto cognitivamente, pois diversos tipos de incertezas tornam difícil a definição e estruturação do problema a ser atacado\cite{Ngo}.

Desta forma, é fundamental para os times de desenvolvimento ágil que o planejamento de release possua um alto nível de confiabilidade\cite{McDaid}.

A disponibilidade de dados confiáveis, na frequência necessária, permite que engenheiros e gerentes tomem decisões que podem ser fundamentais para permitir que o processo de desenvolvimento de software tenha sucesso e que ao final seja entregue um produto de qualidade \cite{codemine}. Processos de desenvolvimento
de software adaptativos são constantemente afetados por mudanças nas condições de negócio. O antigo modo de operação reativo precisa ser substituído por decisões em tempo real e proativas baseadas em informações atualizadas e compreensivas\cite{artAndScience}. Na Microsoft, por exemplo, vários times utilizam dados coletados para melhorar processos como:

\begin{itemize}
    \item Avaliação de riscos e mudança em ferramentas de análise de impacto.
    \item Otimização da estrutura de \textit{branchs} dos repositórios de código.
    \item Análise sócio-técnica de dados.
    \item Busca customizada de bugs e logs.
    \item Acompanhamento de tendências e relatórios do status de desenvolvimento.
\end{itemize}

\textcolor{red}{Os métodos atuais de planejamento de release utilizam de informações, \textit{stakeholders}
e conhecimento estáticos e predefinidos, os principais defeitos dessa abordagem
estão relacionados a incapacidade de lidar com a grande quantidade de dados
relacionados as mudanças constantes que estão acontecendo. A utilização de
\textit{Big Data Analytics} oferece o principal caminho para auxiliar na tomada
de decisões proativas e rápidas para o ciclo de desenvolvimento\cite{artAndScience}.}

\todo[inline, backgroundcolor=yellow!20!white, bordercolor=red]{Reescrever o parágrafo acima. Está bem confuso, além de erros de pontuação. Veja como aqui no texto o mesmo problema da apresentação se manifestou. De repente, do nada, aparecem o Software Analytics e o Big Data. Outro questionamento: por que preciso analisar grandes volumes de dados para planejar release?!?!?}

Neste trabalho serão investigadas técnicas e ferramentas de mineração em repositórios de software para recuperar e analisar dados que possam apoiar a tomada de decisão sobre na atividade de planejamento de releases.


\section{Problema}
\label{int:pro}
Vista a relevância da fase de planejamento de sprint em um projeto ágil,
formulou-se o seguinte problema, que motiva o desenvolvimento desta pesquisa.

\todo[inline, backgroundcolor=yellow!20!white, bordercolor=red]{A relevância da atividade de planejamento de release é sem dúvida uma motivação. Contudo, o problema precisa ser melhor escrito, inclusive com citações que corroborem com a mesma percepção sobre ele}



\begin{center}
 \textit{Dificuldade na tomada de decisões de priorização de features em
  Release Plannings}
\end{center}


\section{Objetivos}
\label{int:obj}
Para uma melhor estruturação do trabalho, foi utilizado o método do
GQM \textit{(Goal, Question, Metric)}, que foi definido pela NASA Goddard Space
Flight, e que propõe uma abordagem topdown para uma melhor organização de objetivos
de um projeto. O GQM propõe uma estrutura em três níveis\cite{gqm}:

\begin{itemize}
    \item \textbf{Objetivos (Goal):} Nível conceitual, onde são definidos os
        objetivos do trabalho ou projeto.
    \item \textbf{Questões (Question):} Nível operacional, onde são definidas um
        conjunto de questões para caracterizar o objeto ou objetivo do projeto.
    \item \textbf{Métricas (Metric):} Nível quantitativo, onde são definidos um
        conjunto de métricas a serem analisadas para que as questões associadas
        a elas possam ser respondidas.
\end{itemize}

\todo[inline, backgroundcolor=yellow!20!white, bordercolor=red]{Como não há validação da solução, então vamos retirar o GQM do TCC1. Descreva o objetivo geral e específicos do trabalho somente. Os objetivos devem ser vistos como o foco, diretrizes que serão persseguidas para que a solução seja construída, bem como a validação.}


Utilizando-se destas ideias foi definido um objetivo geral do trabalho, e alguns
objetivos específicos desta primeira fase. O objetivo geral foi definido como:

\begin{center}
    \textit{Análise do uso de Software Analytics para a priorização
        de atividades em projetos de desenvolvimento ágil.}
\end{center}

Como o trabalho foi dividido em duas fases, os objetivos específicos da primeira
fase foram definidos como:

\begin{itemize}
    \item Levantar fundamentação teórica necessária para apoiar o trabalho.
    \item Definir técnicas e ferramentas de \textit{Software Analytics} que deverão ser utilizadas.
    \item Definir e caracterizar o projeto para estudo de caso.
    \item Montar infraestrutura computacional necessária para a realização do trabalho.
    \item Aplicar técnicas de \textit{Software Analytics} no projeto de estudo de caso e analisar resultados.
\end{itemize}

\section{Questão de Pesquisa}
\label{int:que}
\todo[inline, backgroundcolor=yellow!20!white, bordercolor=red]{Dê uma olhadinha naquele paper da Mary Shaw que dá dicas sobre a formulação da questão de pesquisa}

Diferentemente de outros campos da ciência e engenharia que possuem paradigmas de
pesquisa bem desenvolvidos, a engenharia de software ainda não desenvolveu um
método ou modelo de guia para estes tipos de estudos. Por este motivo, é
comum ver críticas relatórios de pesquisas de software que fogem dos paradigmas
tradicionais\cite{shaw}.A questão de pesquisa é a parte mais importante de um estudo, ela deve guiar toda
a metodologia de pesquisa. Um estudo deve fazer perguntas que que identifiquem e
definam o escopo das atividades de pesquisa\cite{guidelines}.

Com base no problema identificado, foi possível derivar uma questão de pesquisa
que deve ser respondida ao final deste trabalho.

\begin{center}
    \textit{É possível utilizar a relevância das issues de um projeto para ajudar
    na priorização de tarefes de uma sprint?}
\end{center}

\subsection{Questões Secundárias}
\label{int:que:sec}

\todo[inline, backgroundcolor=yellow!20!white, bordercolor=red]{Considere o que conversamos quando da apresentação}


A partir desta questão inicial, é possivel definir questões secundárias para
permitir  a investigação de métodos que possibilitem a solucionar questão primária.
Como o cálculo de relevância pode ser realizado
de formas diferentes é necessário escolher o melhor método para a realização
desta tarefa, além disto métodos de análise devem ser definidos para que
a melhor interpretação dos dados seja feita. As questões secundárias
foram definidas como as seguintes abaixo:

\begin{itemize}
    \item É possível utilizar o algoritmo de ranqueamento de páginas para calcular a relevância das issues?
    \item Quais técnicas de \textit{Software Analytics} podem ser utilizadas para
        determinar se a relevância das issues possuem relação com o planejamento
        de sprint?
    \item É possível utilizar ambientes de Big Data para aumentar o espaço amostral
        do experimento?
\end{itemize}


\section{Organização do Trabalho}
\label{int:org}
Este trabalho foi dividido em cinco capítulos, são eles:

\begin{itemize}
    \item \textbf{Capítulo 1 - Introdução:} Capítulo responsável \todo{O capítulo não é o responsável pela apresentação... Use a voz passiva e reescreva os demais itens} pela apresentação do contexto do trabalho, as ideias que o motivaram, as questões a serem respondidas e objetivos a serem alcançados.
    \item \textbf{Capítulo 2 - Metodologia:} Apresenta-se a metodologia a ser utilizada, tanto de pesquisa quanto de implementação e projeto para o estudo de caso.
    \item \textbf{Capítulo 3 - Referencial Teórico:} Apresenta-se o referencial teórico que serviu como base para essa pesquisa.
    \item \textbf{Capítulo 4 - Projeto de Estudo de Caso:} Descrição da aplicação das técnicas e ferramentas definidas em um projeto de desenvolvimento de software ágil.
    \item \textbf{Capítulo 5 - Conclusão:} Discussão dos resultados e conclusões sobre o estudo de caso e proposta de próximos passos para a continuação do trabalho.
\end{itemize}
