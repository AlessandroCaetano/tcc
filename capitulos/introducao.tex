\chapter{Introdução}
\section{Contexto}
O desenvolvimento de ágil surgiu como uma forma de tomar um outro
caminho em relação ao desenvolvimento convencional software. Ao invés de planejar,
analisar e projetar para um futuro distante, programadores XP fazem todas essas
atividades um pouco de cada vez, durante todo o desenvolvimento\cite{XP}.
Desta forma, quatro princípios foram propostos pela comunidade ágil, tendo Kent
Beck como seu precursor\cite{beckManifesto}:
\begin{itemize}
    \item Indivíduos e interações acima de processos e ferramentas.
    \item Software funcionando acima de uma documentação compreensiva.
    \item Colaboração do cliente acima de negociação de contratos.
    \item Responder as mudanças acima de seguir um plano.
\end{itemize}

No ciclo de desenvolvimento ágil o cliente decide a próxima \textit{release}
escolhendo as \textit{features}, também chamadas de histórias, que agregam
maior valor dentre todas as \textit{features} possíveis, tendo em vista  o custo
e a velocidade de implementação do time. A cada iteração, ou \textit{sprint}, o
cliente escolhe um conjunto de  histórias que devem ser implementadas dentro das
histórias restantes na \textit{release}. A partir disto as histórias são quebradas em
atividades menores que são atribuídas aos programadores, e que depois são
transformadas em casos de testes para provar que tarefa foi concluída\cite{XP}.
Ao final da release, um produto de software é entregue, e o ciclo se inicia
novamente.

O planejamento de release é a atividade mais importante realizada no começo de
cada novo ciclo de desenvolvimento de software no método ágil, é nele que são
definidas as funcionalidades que serão implementadas nos próximos meses e como
os recursos de projeto deverão ser alocados.Mesmo assim, apesar dos níveis de
interação entre pessoas em projetos de desenvolvimento ágil serem altos,
o planejamento de  release é uma tarefa difícil tanto computacionalmente quanto
cognitivamente, pois diversos tipos de incertezas tornam difícil a definição e
estruturação do problema a ser atacado\cite{Ngo}. Desta forma, é fundamental para
os times de desenvolvimento ágil que o planejamento de release possua um alto
nível de confiabilidade\cite{McDaid}.

A disponibilidade de dados confiáveis, na frequência necessária, permite que
engenheiros e gerentes tomem decisões que podem ser fundamentais para permitir
que o processo de desenvolvimento de software tenha sucesso e que ao final seja
entregue um produto de qualidade\cite{codemine}. Processos de desenvolvimento
de software adaptativos são constantemente afetados por mudanças nas condições
de negócio, o antigo modo de operação reativo precisa ser substituído por decisões
em tempo real e proativas baseadas em informações altamente atualizadas e
compreensivas\cite{artAndScience}.Na Microsoft, por exemplo, vários times utilizam
dados coletados para melhorar processos como:

\begin{itemize}
    \item Avaliação de riscos e mudança em ferramentas de análise de impacto.
    \item Otimização da estrutura de \textit{branchs} dos repositórios de código.
    \item Análise sócio-técnica de dados.
    \item Busca customizada de bugs e logs.
    \item Acompanhamento de tendências e relatórios do status de desenvolvimento.
\end{itemize}

Os métodos atuais de planejamento de release utilizam de informações, \textit{stakeholders}
e conhecimento estáticos e predefinidos, os principais defeitos dessa abordagem
estão relacionados a incapacidade de lidar com a grande quantidade de dados
relacionados as mudanças constantes que estão acontecendo. A utilização de
\textit{Big Data Analytics} oferece o principal caminho para auxiliar na tomada
de decisões proativas e rápidas para o ciclo de desenvolvimento\cite{artAndScience}.

A partir do entendimento destes problemas, este estudo utiliza de ferramentas
e técnicas de \textit{Software Analytics} e \textit{Big Data} para avaliar a
utilização de análise de dados para a priorização de atividades em projetos de
desenvolvimento ágil.

\section{Problema}

Vista a relevância da fase de planejamento de sprint em um projeto ágil,
formulou-se o seguinte problema, que motiva o desenvolvimento desta pesquisa.

\begin{center}
 \textit{Dificuldade na tomada de decisões de priorização de features em
  Release Plannings}
\end{center}

\section{Objetivos}

Para uma melhor estruturação do trabalho, foi utilizado o método do
GQM \textit{(Goal, Question, Metric)}, que foi definido pela NASA Goddard Space
Flight, e que propõe uma abordagem topdown para uma melhor organização de objetivos
de um projeto. O GQM propõe uma estrutura em três níveis\cite{gqm}:

\begin{itemize}
    \item \textbf{Objetivos (Goal):} Nível conceitual, onde são definidos os
        objetivos do trabalho ou projeto.
    \item \textbf{Questões (Question):} Nível operacional, onde são definidas um
        conjunto de questões para caracterizar o objeto ou objetivo do projeto.
    \item \textbf{Métricas (Metric):} Nível quantitativo, onde são definidos um
        conjunto de métricas a serem analisadas para que as questões associadas
        a elas possam ser respondidas.
\end{itemize}

Utilizando-se destas ideias foi definido um objetivo geral do trabalho, e alguns
objetivos específicos desta primeira fase. O objetivo geral foi definido como:

\begin{center}
    \textit{Análise do uso de Software Analytics para a priorização
        de atividades em projetos de desenvolvimento ágil.}
\end{center}

Como o trabalho foi dividido em duas fases, os objetivos específicos da primeira
fase foram definidos como:

\begin{itemize}
    \item Levantar fundamentação teórica necessária para apoiar o trabalho.
    \item Definir técnicas e ferramentas de \textit{Software Analytics} que deverão ser utilizadas.
    \item Definir e caracterizar o projeto para estudo de caso.
    \item Montar infraestrutura computacional necessária para a realização do trabalho.
    \item Aplicar técnicas de \textit{Software Analytics} no projeto de estudo de caso e analisar resultados.
\end{itemize}

\section{Questão de Pesquisa}

Diferentemente de outros campos da ciência e engenharia que possuem paradigmas de
pesquisa bem desenvolvidos, a engenharia de software ainda não desenvolveu um
método ou modelo de guia para estes tipos de estudos. Por este motivo, é
comum ver críticas relatórios de pesquisas de software que fogem dos paradigmas
tradicionais\cite{shaw}.A questão de pesquisa é a parte mais importante de um estudo, ela deve guiar toda
a metodologia de pesquisa. Um estudo deve fazer perguntas que que identifiquem e
definam o escopo das atividades de pesquisa\cite{guidelines}.

Com base no problema identificado, foi possível derivar uma questão de pesquisa
que deve ser respondida ao final deste trabalho.

\begin{center}
    \textit{É possível utilizar a relevância das issues de um projeto para ajudar
    na priorização de tarefes de uma sprint?}
\end{center}

\subsection{Questões Secundárias}


\section{Organização do Trabalho}
Este trabalho foi dividido em cinco capítulos, são eles:

\begin{itemize}
    \item \textbf{Capítulo 1 - Introdução:} Capítulo responsável pela apresentação do contexto do trabalho, as ideias que o motivaram, as questões a serem respondidas e objetivos a serem alcançados.
    \item \textbf{Capítulo 2 - Metodologia:} Apresenta-se a metodologia a ser utilizada, tanto de pesquisa quanto de implementação e projeto para o estudo de caso.
    \item \textbf{Capítulo 3 - Referencial Teórico:} Apresenta-se o referencial teórico que serviu como base para essa pesquisa.
    \item \textbf{Capítulo 4 - Projeto de Estudo de Caso:} Descrição da aplicação das técnicas e ferramentas definidas em um projeto de desenvolvimento de software ágil.
    \item \textbf{Capítulo 5 - Conclusão:} Discussão dos resultados e conclusões sobre o estudo de caso e proposta de próximos passos para a continuação do trabalho.
\end{itemize}
