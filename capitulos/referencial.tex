\chapter{Referencial Teórico}
\section{Software Analytics}
Durante muito tempo, a falta de dados em projetos de software foi uma constante.
Agora com o auxílio da internet e dos projetos de software livre, existem tantos
dados relacionados a projetos de software que é manualmente impossível de analisá-los
por completo \cite{artAndScience}. Ao fim de 2012, pesquisas mostram que o \textit{Mozilla Firefox} 
teve 800.000 bug reports e outras plataformas como \textit{Sourcefoge.net}
e o \textit{Github} hospedam 324.000 e 11.2 milhões de projetos, respectivamente \cite{informationNeeds}.

Para ser capaz de manipular essa grande quantidade de dados, muitos pesquisadores
se voltaram para o uso e \textit{analytics}, ou seja, o uso de análises, dados e
raciocínio sistemático para tomar decisões. Podemos definir \textit{Software Analytics}
como: 'A análise de dados de software para gerentes e engenheiros de software, 
com o objetivo de capacitar individuos e times de desenvolvimento, a ganhar e difundir 
conhecimento a partir de seus dados para tomar melhores decisões'\cite{informationNeeds}.

Hoje é comum empresas como Google, Facebook e Microsoft aplicarem métodos de data
science diariamente em seus projetos. Além disto, o número de conferencias neste
tópico aumentou bastante, sendo as duas mais importantes \textit{Mining Software
Repositories} (MSR) e a \textit{PROMISE Conference on Repeatable Experimentes in Software
Engineering}, cada uma com um foco diferente, sendo a MSR procupada com a coleta dos dados
enquanto a PROMISE com a eficacia e repetibilidade da análise de dados.

\section{Centralidade de Redes}

Centralidade é um conceito fundamental e um dos tópicos mais estudados na análise 
de redes sociais, ele determina o grau de importancia de um vértice dentre todos
os outros dentro de uma rede. O conceito de centralidade de redes já foi aplicado a diversos contextos,
dentre eles: investigar a influencia de redes interorganizacionais, estudos de relevância,
vantagens em redes de troca, competencia em organizações formais, oportunidades de emprego 
e diversos outros campos do mercado e ciência\{centrality}.

\subsection{Page Ranking}
