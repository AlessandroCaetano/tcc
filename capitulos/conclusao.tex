\chapter{Considerações Preliminares e Trabalhos Futuros}

Neste trabalho, foi apresentado o projeto de pesquisa a ser realizado e os conceitos fundamentais que o rodeiam, como o problema a resolvido, a questão de pesquisa e a estruturação do exemplo de uso. A primeira etapa deste trabalho apresentou uma revisão bibliográfica a respeito de técnicas e ferramentas utilizadas com \textit{Software Analytics}, aplicações do algoritmo de \textit{Page Ranking}. Além disto, a revisão bibliográfica realizada englobou a atividade de planejamento de \textit{release}, de forma a elucidar como esta atividade é afetada pelas incertezas de um projeto e buscar metodologias propostas por autores diversos em como mitigar este problema. 

Os conceitos levantados na revisão bibliográfica foram então utilizados para a implementação de um exemplo de uso, onde foram aplicadas algumas das técnicas de \textit{Software Analytics} e o algoritmo de ranqueamento de páginas com o intuito de relacionar a relevância das \textit{issues} do projeto com o planejamento de \textit{release}. Para a realização desta tarefa foram utilizadas as tecnologias apresentadas na Seção~\ref{met} e foi escolhido um projeto de software livre que faz parte da população proposta neste estudo, o projeto utilizado para o exemplo de uso foi o Software Público Brasileiro, e nele, foram analisadas 879 \textit{issues} presentes no repositório do projeto. Utilizando o ranque gerado, foi realizada uma comparação manual das  \textit{issues} com a priorização das \textit{features} presentes no repositório do SPB.

A segunda etapa deste trabalho será responsável por prosseguir com o levantamento bibliográfico necessário, bem como estruturar um estudo de caso de forma a validar as proposições feitas nesta primeira etapa. Nesta segunda parte também deverá ser evoluído o método de relacionamento das \textit{issues} e \textit{features} de um repositório, de forma a automatizar como essas informações estão sendo relacionadas. Com isto, será possível validar a solução proposta e com isto responder a questão de pesquisa definida nesta primeira etapa.
\newpage
\section{Cronograma}

Esta seção contem o cronograma de atividades para o TCC 2. As atividades foram criadas com o objetivo de cumprir com os objetivos propostos neste trabalho e estão passiveis de mudanças nas datas caso algum atraso ou dificuldade seja encontrada durante a realização das atividades.

\begin{table}[h]
    \begin{tabularx}{\textwidth}{|X|r|}
        \toprule
    	\textbf{Atividade} & Data  \\
    	\hline	
            \midrule
            Pesquisa de referencial teórico a respeito dos tópicos de \textit{Big Data} & 10/08 a 20/08 \\ \hline
            Definição de ferramentas de análise e \textit{Big Data} & 20/08 a 10/09 \\ \hline
            Seleção dos repositórios do Github para serem analisados & 10/09 a 25/09 \\ \hline
            Alterações no texto para adequação a um estudo de caso & 25/09 a 10/10 \\ \hline
            Implementação da infraestrutura computacional para a realização do estudo de caso & 10/10 a 20/10 \\ \hline
            Execução do estudo de caso em ambiente de \textit{Big Data} & 20/10 a 01/11 \\ \hline
            Escrita dos Resultados e Considerações Finais do TCC 2 & 01/11 a 20/11 \\ \hline
            \bottomrule
    \end{tabularx}
    \label{tab:crono}
    \caption{Cronograma Para o TCC 2}
\end{table}


