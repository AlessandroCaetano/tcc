\chapter{Considerações Finais e Trabalhos Futuros}

Neste trabalho, foi apresentado o projeto de pesquisa a ser realizado e os conceitos fundamentais que o rodeiam, como o problema a resolvido, a questão de pesquisa e a estruturação do exemplo de uso. A primeira etapa deste trabalho apresentou uma revisão bibliográfica a respeito de técnicas e ferramentas utilizadas com \textit{Software Analytics}, aplicações do algoritmo de \textit{Page Ranking}, e a cerca da definição de como deve ser realizado um estudo de caso. Além disto, a revisão bibliográfica realizada englobou a atividade de planejamento de \textit{release}, de forma a elucidar como esta atividade é afetada pelas incertezas de um projeto e buscar metodologias propostas por autores diversos em como mitigar este problema. 

Os conceitos levantados na revisão bibliográfica foram então utilizados para a implementação de um exemplo de uso, onde foram aplicadas algumas das técnicas de \textit{Software Analytics} e o algoritmo de ranqueamento de páginas com o intuito de relacionar a relevância das \textit{issues} do projeto com o planejamento de \textit{release}. O projeto utilizado para o exemplo de uso foi o Software Público Brasileiro, e nele, foram analisadas 879 \textit{issues} presentes no repositório do projeto. Utilizando o ranque gerado, foi realizada uma comparação manual das  \textit{issues} com a priorização das \textit{features} presentes no repositório do SPB.

A segunda etapa deste trabalho será responsável por prosseguir com o levantamento bibliográfico necessário, bem como estruturar um estudo de caso de forma a validar as proposições feitas nesta primeira etapa. Nesta segunda parte também deverá ser evoluído o método de relacionamento das \textit{issues} e \textit{features} de um repositório, de forma a automatizar como essas informações estão sendo relacionadas. Com isto, será possível validar a solução proposta e com isto responder a questão de pesquisa definida nesta primeira etapa.
