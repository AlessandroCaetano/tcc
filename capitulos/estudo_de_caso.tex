\chapter{Estudo de Caso}
\label{est}
\section{Software Público Brasileiro}
\label{est:sof}

O portal do Software Público Brasileiro (SPB) é um espaço virtual, criado em 12 de 
abril de 2007, que agrega um conjunto de softwares desenvolvidos nas áreas de saúde, 
educação, saneamento, gestão de TI, TV Digital e Geoprocessamento. Os serviços do
SPB são acessados não só no Brasil, mas também por outros países como por exemplo:
Uruguai, Argentina, Portugal, Venezuela, Chile e Paraguai. A politica de compartilhamento
de software resulta em uma gestão de recursos e gastos de informática mais racionalizada,
ampliação de parcerias e reforço da politica de software livre no setor público~\cite{spb}.

O novo Portal do Software Público Brasileiro é composto de:

\begin{itemize}
    \item Lista de discussão (Mailman).
    \item Plataforma de redes sociais com blog, e-Portifólios, RSS, discussão 
    temática, agenda de eventos, galeria de imagens, e demais funcionalidades de um CMS (Noosfero).
    \item Sistema de controle de versão, repositório de código-fonte e ambiente 
    desenvolvimento colaborativo (GitLab).
    \item Autenticação única, busca e integração de ferramentas a fim de tornar 
    a navegação intuitiva e transparente entre as diversas ferramentas que compõem 
    o novo Portal (Colab).
\end{itemize}

\subsection{Metodologia de Desenvolvimento}
\label{est:sof:met}

O Software Público Brasileiro foi desenvolvido com base no Scrum. O
Scrum é um framework para o desenvolvimento e manutenção de produtos usado desde
o incio dos anos 90 que possui a característica de permitir que vários processos
e tecnicas possam ser empregadas juntamente a ele. Os três valores essenciais do
Scrum são:

\begin{itemize}
    \item Leveza.
    \item Simples de entender.
    \item Difícil de dominar.
\end{itemize}

O Scrum é fundamentado nas teorias empíricas de controle de processo, ou seja,
utiliza o conhecimento vindo de experiencias e das decisões passadas. Por isto, este 
framework emprega uma abordagem iterativa e incremental para permitir aperfeiçoar
a previsibilidade e os riscos. No Scrum, cada componente é essencial
para o sucesso, por isso existem regras especificas para os times, artefatos e
papéis dentro do framework.


\subsubsection{Práticas do Scrum}
\label{est:sof:met:pra}

O Scrum define algumas práticas que são aplicadas durante o desenvolvimento com 
o objetivo de minimizar a necessidade de reuniões. Estas práticas podem ser eventos
com uma duração definida e buscam aumentar as interações entre as pessoas. A lista
abaixo apresenta esses eventos e suas respectivas descrições.

\begin{itemize}
    \item \textbf{Sprint:} Um intervalo de tempo em que são desenvolvidos os itens
        propostos. Este intervalo geralmente é de duas a quatro semanas e ao final
        de cada sprint, um incremento de software é entregue.
    \item \textbf{Sprint Review:} Ao final de cada sprint o time revisa as atividades
        realizadas durante o período em que a sprint ocorreu. São levantados pontos de
        melhoria e quais foram os pontos fortes do time.
    \item \textbf{Sprint Planning:} No inicio de cada sprint, é feita uma reunião em
        que são priorizadas as atividades que serão realizadas.
    \item \textbf{Product Backlog:} Conjunto de itens que serão desenvolvidos no projeto.
        A cada sprint um conjunto de itens são escolhidos, retirados do \textit{Product
        Backlog} e implementados.
\end{itemize}

\subsection{Organização do Trabalho}
\label{est:sof:org}

\section{Ranqueamento de Páginas no SPB}
\label{est:ran}

\subsection{Nós e Links}
\label{est:ran:nos}

\subsection{Visão Geral}
\label{est:ran:vis}

\section{Resultados}
\label{est:res}
