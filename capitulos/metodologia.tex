\section{Metodologia de Pesquisa}

\section{Metodologia de Busca}
Enquanto as técnicas de pesquisa em engenharia de software
ganham maturidade, muitos pesquisadores sugerem o uso de frameworks e metodologias
vindas de outras área de conhecimento, como a medicina\cite{guidelines}.Recentemente
Petticrew e Roberts sugeriram a utilização do modelo PICOC \textit{(Population,
Intervention, Comparison, Outcome, Context)}, que é divido da seguinte forma\cite{petticrew}:

\begin{itemize}
    \item \textbf{População (Population):} A população que o experimento será
        aplicado, no contexto de engenharia de software pode ser: uma determinada
        função (Desenvolvedores, Gerentes, Usuários), categoria de engenheiro de
        software (Iniciante,Junior,Senior), uma área de aplicação, ou um setor
        da industria.
    \item \textbf{Intervention (Intervenção):} É a metodologia, ferramenta,
        tecnologia ou procedimento aplicado no experimento.
    \item \textbf{Comparação (Comparison):} É a metodologia, ferramenta,
        tecnologia ou procedimento com o qual a intervenção está sendo comparada.
    \item \textbf{Resultado  (Outcome):} Resultados da aplicação da intervenção na
        população.
    \item \textbf{Contexto (Context):} Contexto onde o experimento foi realizado
        (Academia ou Industria), os participantes que fizeram parte do experimento
        (Alunos, Professores, Profissionais), e as tarefas que foram feitas
        (Grande ou Baixa Escala).
\end{itemize}

Utilizando o modelo PICOC é possível estruturar a questão acima da seguinte forma:

\begin{itemize}
    \item \textbf{População (Population):} Repositórios de código aberto.
    \item \textbf{Intervention (Intervenção):} Análise de dados em ambientes de
        \textit{Big Data} com utilização de algoritmos \textit{Page Ranking} e
        centralidade de redes em \textit{issues} de repositórios de código.
    \item \textbf{Comparação (Comparison):} Planejamentos de release sem a utilização
        sistemática de análise de dados.
    \item \textbf{Resultado  (Outcome):} Determinar a eficiência da utilização de
        \textit{Software Analytics} para apoiar decisões de planejamento de release.
    \item \textbf{Contexto (Context):} Projetos de software livre que utilizam
        repositórios de código distribuídos e que fazem a organização das releases
        utilizando \textit{issues}.
\end{itemize}

\section{Estudo de Caso}
