\chapter{Metodologia}
\label{met}

A utilização de experimentação é relativamente recente na área de Engenharia de Software. Um dos trabalhos seminais nessa área foi proposto por \citeonline{guidelines-2007}, e teve o objetivo de estabelecer um guia para se realizar revisões sistemáticas da literatura. Nesta monografia foram utilizados alguns elementos da estrutura do protocolo proposto por \citeonline{guidelines-2007} de forma a sistematizar a busca por estudos que orientassem esse trabalho. Um dos elementos previstos no protocolo, e que foi utilizado nesta monografia, é a composição de uma \textit{string} de busca em bases digitais, originalmente proposto na área de medicina, o PICO. \cite{citeulike:10757239} Trata-se de uma abstração proposta para agrupar conjuntos de termos e auxiliar a composição da \textit{string} de busca a ser submetida às bases. Segundo a adaptação proposta, o PICO procura organizar:

\begin{itemize}
    \item \textit{Population} (População): A população que o estudo será aplicado, no contexto de engenharia de software pode ser: uma determinada função (Desenvolvedores, Gerentes, Usuários), categoria de engenheiro de software (Iniciante, Junior, Senior), uma área de aplicação, ou um setor da industria.
    \item \textit{Intervention} (Intervenção): É a metodologia, ferramenta, tecnologia ou procedimento aplicado no estudo.
    \item \textit{Comparison} (Comparação): É a metodologia, ferramenta, tecnologia ou procedimento com o qual a intervenção está sendo comparada.
    \item \textit{Outcome} (Resultado): Resultados da aplicação da intervenção na população.
\end{itemize}

Considerando os objetivos e as questões de pesquisa apresentadas, na introdução desta monografia, a estrutura do PICO para este trabalho foi assim definida:

\begin{itemize}
    \item \textbf{População:} Repositórios de software-livre.
    \item \textbf{Intervention (Intervenção):} Busca, mineração e análise de dados com utilização do algoritmo de \textit{Page Ranking} em \textit{issues} de repositórios de código.
    \item \textbf{Comparação:} Não se aplica a este estudo pois o exemplo de uso construído não visa a comparação com alguma técnica já conhecida de priorização de atividades.
    \item \textbf{Resultado:} Determinar a eficiência da utilização de algumas das técnicas de \textit{Software Analytics} de forma a apoiar decisões de planejamento de \textit{release}.
\end{itemize}

\section{Definição da População}
\label{met:def}
A população deste estudo foi definida como repositórios de software-livre, mais especificamente aqueles repositórios hospedados na plataforma Github. Como foi discutido na Seção~\ref{ref:sof:sl} esta plataforma, apesar de não ser livre, é a mais popular do mundo e hoje hospeda cerca de 38 milhões de projetos de software. Além disto, outros estudos como o realizado por~\citeonline{githubPop} mostram como essa plataforma pode ser utilizada para a busca e mineração de informação.

Além da popularidade dominante do Github em relação a outras plataformas baseadas em Git, outro dos fatores que direcionaram a escolha desta ferramenta como população é a existência da API também discutida na Seção~\ref{ref:sof:sl}. Através desta solução é possível extrair informações diversas dos repositórios no Github, entre elas podemos citar: \textit{commits}, contribuidores, comentários, trechos de código, \textit{wiki}, \textit{milestones} e \textit{issues}. Esta API também torna possível a extração de dados do repositório como número de estrelas concedidas por outros desenvolvedores, número de \textit{forks} e qual a linguagem em que o projeto está sendo desenvolvido.

Como a API do Github pode fornecer informações de \textit{issues}, \textit{milestones} e da \textit{wiki} de um repositório, e esta monografia deve se limitar a analisar estas informações, ela deve ser utilizada para isto. Uma observação que é necessário fazer é que a API tem a capacidade de retornar no máximo 100 eventos por requisição, ou página, e possui um limite e 400 páginas, desta forma deverão ser analisados repositório que possuam um máximo de 40000 \textit{issues}, sendo que cada \textit{issue} deve possuir um número máximo de 40000 comentários.

\section{Método de Intervenção}
\label{met:int}
A intervenção é a busca, mineração e análise das \textit{issues} dos repositórios. Para a realização desta tarefa foi selecionada a linguagem de programação Python.O Python é uma linguagem de programação dinâmica, interpretada, de alto nível e de  uso geral criada por Guido Van Rossum. Esta linguagem possui estruturas de dados de alto nível, é dinamicamente tipada e é muito atrativa para o desenvolvimento rápido de aplicações e scripts~\cite{python}.

O Python foi escolhido como linguagem principal para o desenvolvimento da solução proposta pela quantidade de bibliotecas e ferramentas disponíveis para trabalhar com requisições web, busca e análise textual e criação, análise e visualização de grafos. A utilização destas ferramentas tem o objetivo de facilitar as análises realizadas neste estudo de forma a reduzir o trabalho de implementação de algoritmos que já são extremamente usados e estudados. Dentre as bibliotecas pesquisadas, as principais e que foram utilizadas neste trabalho foram:

\begin{itemize}
    \item \textbf{Jellyfish:} Biblioteca Python para a realização de correspondência aproximada de textos de palavras, podendo realizar a comparação entre distancia de palavras com utilização de algoritmos como os de: Levenshtein, Jaro-Winkler e Hamming.
    \item \textbf{NetworkX:} Pacote Python para a criação, manipulação e estudo de estruturas, dinâmicas e funções de redes complexas. Pode ser utilizada para geração de grafos randômicos e sintéticos e possui diversos algoritmos para análise de redes como o algoritmo de ranqueamento de páginas e o Hits.
    \item \textbf{NLTK:} Abreviação para \textit{Natural Languagem Toolkit}, é uma plataforma para a construção de programas Python que trabalhem com dados compostos por linguagem humana. Possui diversas ferramentas para processamento de texto em linguagens naturais, oferecendo suporte para o português e o inglês.
\end{itemize}

É importante citar que todas essas bibliotecas e ferramentas utilizadas são livres, de forma que o código fonte de cada uma delas pode ser baixado, executado e alterado para a utilização neste estudo.

Utilizando estas ferramentas foi criado o \textit{script} descrito no apêndice~\ref{ape:cod}. Este \textit{script} realiza a função de buscar no repositório definido as \textit{issues} e os comentários referentes a elas. Depois da execução deste primeiro passo, é realizada uma análise textual para determinar se uma \textit{issue} foi mencionada nos comentários de outra. Desta forma, foi gerado um grafo direcionado onde para cada marcação que uma \textit{issue} possui era adicionada uma aresta para ela. Assim que o grafo foi gerado, utilizou-se o algoritmo de ranqueamento de páginas para determinar quais vértices, representados pelas \textit{issues}, são os mais revelantes.

\section{Analise dos Resultados}
\label{met:res}

Nesta primeira etapa do trabalho a analise dos resultados foi realizada manualmente, porém espera-se que para a etapa seguinte seja possível realizar esta analise de forma automatizada utilizando a biblioteca \textit{Jellyfish} apresentada na Seção~\ref{met:int}.

Para que fosse possível analisar os dados gerados pela intervenção foi feita a comparação manual da relevância das \textit{issues} analisadas e o planejamento de \textit{release} presente na \textit{Wiki} do projeto escolhido. Isso foi possível pois o projeto do Software Público Brasileiro, ou SPB, possui a característica de ter organizado todo o planejamento de suas duas últimas \textit{releases}, ou seja a \textit{release} 4 e 5, inteiramente dentro do repositório de código.

Sendo assim, foi possível alinhar o texto das \textit{issues} mais relevantes com as \textit{features} consideradas estratégicas para estas \textit{releases} e desta forma determinar se a intervenção teve resultados satisfatórios.

\section{Considerações Finais do Capitulo}
Neste capitulo foram apresentadas as principais estruturas, técnicas e ferramentas utilizados para a elaboração desta monografia. Primeiramente foi apresentada a metodologia de pesquisa e a identificação da população de pesquisa, o método de intervenção e de análise de resultados. Logo após cada um destes pontos é discutido, de forma a justificar a escolha dos repositórios do Github como população, a utilização de busca, mineração e análise dos dados dos repositórios utilizando o Python, e como a análise dos resultados foi feita nesta primeira etapa do trabalho. No próximo capitulo será apresentado como estas técnicas foram utilizadas para que fosse possível executar o exemplo de uso que faz parte desta monografia e alguns resultados colhidos a partir deste exemplo.


