\chapter{Metodologia}
\label{met}
\section{Metodologia de Pesquisa}
\label{met:pes}
Enquanto as técnicas de pesquisa em engenharia de software
ganham maturidade, muitos pesquisadores sugerem o uso de frameworks e metodologias
vindas de outras área de conhecimento, como a medicina\cite{guidelines}.Recentemente
Petticrew e Roberts sugeriram a utilização do modelo PICOC \textit{(Population,
Intervention, Comparison, Outcome, Context)}, que é divido da seguinte forma\cite{petticrew}:

\begin{itemize}
    \item \textbf{População (Population):} A população que o experimento será
        aplicado, no contexto de engenharia de software pode ser: uma determinada
        função (Desenvolvedores, Gerentes, Usuários), categoria de engenheiro de
        software (Iniciante,Junior,Senior), uma área de aplicação, ou um setor
        da industria.
    \item \textbf{Intervention (Intervenção):} É a metodologia, ferramenta,
        tecnologia ou procedimento aplicado no experimento.
    \item \textbf{Comparação (Comparison):} É a metodologia, ferramenta,
        tecnologia ou procedimento com o qual a intervenção está sendo comparada.
    \item \textbf{Resultado  (Outcome):} Resultados da aplicação da intervenção na
        população.
    \item \textbf{Contexto (Context):} Contexto onde o experimento foi realizado
        (Academia ou Industria), os participantes que fizeram parte do experimento
        (Alunos, Professores, Profissionais), e as tarefas que foram feitas
        (Grande ou Baixa Escala).
\end{itemize}

Utilizando o modelo PICOC é possível estruturar os termos relativos a esta pesquisa
da seguinte forma:

\begin{itemize}
    \item \textbf{População (Population):} Repositórios de código aberto.
    \item \textbf{Intervention (Intervenção):} Análise de dados em ambientes de
        \textit{Big Data} com utilização de algoritmos \textit{Page Ranking} e
        centralidade de redes em \textit{issues} de repositórios de código.
    \item \textbf{Comparação (Comparison):} Planejamentos de release sem a utilização
        sistemática de análise de dados.
    \item \textbf{Resultado  (Outcome):} Determinar a eficiência da utilização de
        \textit{Software Analytics} para apoiar decisões de planejamento de release.
    \item \textbf{Contexto (Context):} Projetos de software livre que utilizam
        repositórios de código distribuídos e que fazem a organização das releases
        utilizando \textit{issues}.
\end{itemize}


A partir da estrutura apresentada acima, que utiliza o método PICOC, podemos agrupar
os termos relevantes a esta pesquisa em uma \textit{string} onde os termos da mesma
família são agrupados utilizando o operador \textbf{AND} e as familias são agrupadas
com o operador \textbf{OR}. Desta forma, foi possivel estruturar uma \textit{string} 
de busca que permitisse abranger uma maior quantidade de \textit{papers} relevantes 
para esta pesquisa. Esta string foi definida como:

\begin{center}
    \textit{(GIT AND GITHUB AND GITLAB AND REPO AND REPOSITORIES AND FREE
    SOFTWARE AND SOFTWARE) OR (DATA ANALYTICS AND SOFTWARE ANALYTICS AND BIG DATA
    AND DATA MINING AND PAGE RANKING AND CENTRALITY AND GRAPH THEORY AND GRAPH AND
    FLOW NETWORKS AND NETWORK) OR (RELEASE AND RELEASE PLANNING AND PLANNING AND
    AGILE METHODS AND WIKI AND ISSUE AND AGILE AND PRIORITIZE) OR (EFFICIENCY AND 
    APPROACH AND THEORETICAL AND RESULT)}
\end{center}

Esta \textit{string} foi aplicada a base Scopus e todos os \textit{papers} 
utilizados como base para este trabalho foram retirados do conjunto de resultados
retornados.

\section{Tecnologias Utilizadas}
\label{met:tec}
Para a realização deste trabalho, foram escolhidas as ferramentas e serviços
que serão utilizadas para a gerencia de atividades, repositórios de código e
texto e desenvolvimento. As escolhas foram feitas com base na experiência de uso
destas ferramentas.

\subsection{Git}
\label{met:tec:git}
Git é uma ferramenta de controle de versão gratuita e de código aberto criada
por Linus Torvalds em 2005~\cite{chacon}. Neste trabalho o Git foi utilizado 
para controle de versão dos códigos utilizados na implementação do trabalho
e deste texto.

\subsection{Github}
\label{met:tec:github}
O Github é um serviço que provê a utilização do Git em uma plataforma Web. Ele 
disponibiliza repositórios remotos para armazenamento de código, wikis e serviços
de \textit{tracking} de issues~\cite{github}. Além disto, o Github ainda possui uma API para
que permite alterar e consultar repositórios utilizando requisições HTTP.

\subsection{Gitlab}
\label{met:tec:gitlab}
O Gitlab, assim como o Github, é um serviço que provê a utilização do Git como
uma plataforma Web. Ele também possui repositórios remotos para armazenamento
de código, wikis, serviços de \textit{tracking de issues}. Porém diferentemente
do Github, o Gitlab é um software livre licenciado sob a licença MIT~\cite{gitlab}.

\subsection{Python}
\label{met:tec:python}
Python é uma linguagem de programação dinâmica, interpretada, de alto nível e de 
uso geral criada por Guido Van Rossum. Esta linguagem possui estruturas de dados
de alto nível, é dinamicamente tipada e é muito atrativa para o desenvolvimento rápido
de aplicações e scripts~\cite{python}. Neste trabalho, a linguagem Python foi utilizada para a criação
de scripts e a execução do algoritmo de ranqueamento de páginas.

\section{Estudo de Caso}
\label{met:est}
A estratégia de pesquisa escolhida para que fosse aplicada neste trabalho foi o
estudo de caso. Um estudo de caso é uma investigação empírica que investiga um 
fenômeno contemporâneo dentro de seu contexto da vida real, especialmente quando 
os limites entre o fenômeno e o contexto não estão claramente definidos. Um dos 
grandes motivos para a escolha desta técnica é que ela não requer controle sobre
os eventos por parte do pesquisador~\cite{yin}.

O primeiro componente importante para a elaboração de um estudo de caso é a definição
de uma questão de pesquisa, o que foi feito na sessão~\ref{int:que}. O segundo
componente necessário é uma proposição de estudo que deve ser feita com base na questão 
de pesquisa para que seja possível responde-la, no caso deste trabalho é como a utilização de \textit{Software 
Analytics} pode ajudar nas tomadas de decisões de priorização em \textit{Release Plannings}.
O terceiro ponto do estudo de caso é a unidade de análise, ou seja a definição de 'caso',
que, para este trabalho, é um repositório de código que possua uma estrutura organizacional
dentro da Wiki e das \textit{issues} do próprio repositório~\cite{yin}.

Os próximos componentes do estudo de caso estão relacionados a como os dados das
proposições estão unidos e quais são critérios de interpretação das descobertas.
Neste estudo, são relacionados dados da Wiki, dos \textit{milestones}, das \textit{issues} 
e dos comentários dos repositórios de código com a relevância das \textit{issues}, para isto
é utilizado como parâmetro o ranque das \textit{issues} para determinar quais delas 
são mais importantes dentro do espaço amostral.

Desta forma, podemos definir os componentes do estudo de caso como os seguintes:

\begin{itemize}
    \item \textbf{Questão de Pesquisa:} Questão que baseia o estudo, definida em~\ref{int:que}.
    \item \textbf{Proposições:} Como a utilização de \textit{Software Analytics} pode ajudar nas
    nas tomadas de decisões de priorização de \textit{Release Planning}.
    \item \textbf{Unidade de Análise:} Neste trabalho, um caso é definido como um repositório
        de código cuja organização é feita dentro do repositório nas Wikis, \textit{milestones}
        e \textit{issues}.
\end{itemize}


