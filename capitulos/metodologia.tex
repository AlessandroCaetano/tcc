\chapter{Metodologia}
\label{met}
\section{Metodologia de Pesquisa}
\label{met:pes}


A utilização de experimentação é relivamente recente na área de Engenharia de Software. Um dos trabalhos seminais nessa área foi proposto por \citeonline{guidelines-2007}, e teve o objetivo de estabelecer um guia para se realizar revisões sistemáticas da literatura. Nesta monografia foram utilizados alguns elementos da estrutura do protocolo proposto por \citeonline{guidelines-2007} de forma a sistematizar a busca por estudos que orientassem esse trabalho. Um dos elementos previstos no protocolo, e que foi utilizado nesta monografia, é a composição de uma \textit{string} de busca em bases digitais, originalmente proposto na área de medicina, o PICO. \cite{citeulike:10757239} Trata-se de uma abstratação proposta para agrupar conjuntos de termos e auxiliar a composição da \textit{string} de busca a ser submetida às bases. Segundo a adaptação proposta, o PICO procura organizar:



\begin{itemize}
    \item \textit{Population} (População): A população que o estudo será aplicado, no contexto de engenharia de software pode ser: uma determinada função (Desenvolvedores, Gerentes, Usuários), categoria de engenheiro de software (Iniciante, Junior, Senior), uma área de aplicação, ou um setor da industria.
    \item \textit{Intervention} (Intervenção): É a metodologia, ferramenta, tecnologia ou procedimento aplicado no estudo.
    \item \textit{Comparison} (Comparação): É a metodologia, ferramenta, tecnologia ou procedimento com o qual a intervenção está sendo comparada.
    \item \textit{Outcome} (Resultado): Resultados da aplicação da intervenção na população.
\end{itemize}



Assim, neste trabalho os termos da \textit{string} de busca foram assim definidos:

\todo[inline, backgroundcolor=yellow!20!white, bordercolor=red]{Reescrever o PICO e a STRING de busca de acordo com nossas conversas}

\begin{itemize}
    \item \textbf{População:} Repositórios de software-livre.
    \item \textbf{Intervention (Intervenção):} Análise de dados em ambientes de \textit{Big Data} com utilização de algoritmos \textit{Page Ranking} e centralidade de redes em \textit{issues} de repositórios de código.
    \item \textbf{Comparação} Não procede.
    \item \textbf{Resultado:} Determinar a eficiência da utilização de \textit{Software Analytics} para apoiar decisões de planejamento de release.
\end{itemize}


De acordo com a estrutura do PICO, os termos que pertencem ao mesmo grupo/conjuntos são agrupados utilizando o operador \textbf{OR} e entre grupos/conjuntos, são agrupadas com o operador \textbf{AND}. Desta forma, a string de busca deste trabalho foi definida como:

\begin{center}
    \textit{(GIT AND GITHUB AND GITLAB AND REPO AND REPOSITORIES AND FREE
    SOFTWARE AND SOFTWARE) OR (DATA ANALYTICS AND SOFTWARE ANALYTICS AND BIG DATA
    AND DATA MINING AND PAGE RANKING AND CENTRALITY AND GRAPH THEORY AND GRAPH AND
    FLOW NETWORKS AND NETWORK) OR (RELEASE AND RELEASE PLANNING AND PLANNING AND
    AGILE METHODS AND WIKI AND ISSUE AND AGILE AND PRIORITIZE) OR (EFFICIENCY AND 
    APPROACH AND THEORETICAL AND RESULT)}
\end{center}

A base digital escolhida foi a Scopus. A \textit{string} foi aplicada e os principais artigos que norteaream esta monografia foram recuperados do conjunto de resultados retornado.

\section{Tecnologias Utilizadas}
\label{met:tec}
Para a realização deste trabalho, foram escolhidas as ferramentas e serviços
que serão utilizadas para a gerencia de atividades, repositórios de código e
texto e desenvolvimento. As escolhas foram feitas com base na experiência de uso
destas ferramentas.

\subsection{Git}
\label{met:tec:git}
Git é uma ferramenta de controle de versão gratuita e de código aberto criada
por Linus Torvalds em 2005~\cite{chacon}. Neste trabalho o Git foi utilizado 
para controle de versão dos códigos utilizados na implementação do trabalho
e deste texto.

\subsection{Github}
\label{met:tec:github}
O Github é um serviço que provê a utilização do Git em uma plataforma Web. Ele 
disponibiliza repositórios remotos para armazenamento de código, wikis e serviços
de \textit{tracking} de issues~\cite{github}. Além disto, o Github ainda possui uma API para
que permite alterar e consultar repositórios utilizando requisições HTTP.

\subsection{Gitlab}
\label{met:tec:gitlab}
O Gitlab, assim como o Github, é um serviço que provê a utilização do Git como
uma plataforma Web. Ele também possui repositórios remotos para armazenamento
de código, wikis, serviços de \textit{tracking de issues}. Porém diferentemente
do Github, o Gitlab é um software livre licenciado sob a licença MIT~\cite{gitlab}.

\subsection{Python}
\label{met:tec:python}
Python é uma linguagem de programação dinâmica, interpretada, de alto nível e de 
uso geral criada por Guido Van Rossum. Esta linguagem possui estruturas de dados
de alto nível, é dinamicamente tipada e é muito atrativa para o desenvolvimento rápido
de aplicações e scripts~\cite{python}. Neste trabalho, a linguagem Python foi utilizada para a criação
de scripts e a execução do algoritmo de ranqueamento de páginas.



\todo[inline, backgroundcolor=yellow!20!white, bordercolor=red]{Pensei em aproveitar a seção abaixo, mas entendo que seria muita forçação de barra. Pensando bem é melhor retirá-la, ou então reescrever o texto, ao moldes do que eu fiz no PICO, e dizendo que nos inspiramos na estrutura proposta para um estudo de caso. Lembre-se que no TCC2 estruturaremos o estudo de caso como porposto por Yin. Caso você decida retirar é preciso descrever a idéia de um exemplo de uso que será descrito no cap. 4. Existem outros aspectos relacionados a metodologia que você poderia utilizar e enriquecer esse capítulo, como naquele paper que te passei, o Towards a decision-making structure for selecting a research design in empirical software engineering.}

\section{Estudo de Caso}
\label{met:est}
O método de pesquisa escolhido para este trabalho foi o estudo de caso. Um estudo de caso é uma investigação experimental que investiga um fenômeno contemporâneo dentro de seu contexto da vida 
os limites entre o fenômeno e o contexto não estão claramente 
grandes motivos para a escolha deste método é que ele não requer controle sobre os eventos por parte do pesquisador~\cite{yin}.

O primeiro componente importante para a elaboração de um estudo de caso é a definição de uma questão de pesquisa, conforme definida na sessão~\ref{int:que}. O segundo componente necessário é uma proposição de estudo que deve ser feita com base na questão 
de pesquisa para que seja possível responde-la, no caso deste trabalho é como \textcolor{red}{ REESCREVER --
 a utilização de \textit{Software 
Analytics} pode ajudar nas tomadas de decisões de priorização em \textit{Release Plannings}.} O terceiro ponto do estudo de caso é a unidade de análise, ou seja a definição do 'caso',
que, para este trabalho, é o SBP.

Os próximos componentes do estudo de caso estão relacionados a como os dados das
proposições estão unidos e quais são critérios de interpretação das descobertas.
Neste estudo, são relacionados dados da Wiki, dos \textit{milestones}, das \textit{issues} 
e dos comentários dos repositórios de código com a relevância das \textit{issues}, para isto
é utilizado como parâmetro o ranque das \textit{issues} para determinar quais delas 
são mais importantes dentro do espaço amostral.

\todo[inline, backgroundcolor=yellow!20!white, bordercolor=red]{Reescrever o PICO e a STRING de busca de acordo com nossas conversas}



Desta forma, podemos definir os componentes do estudo de caso como os seguintes:

\begin{itemize}
    \item \textbf{Questão de Pesquisa:} Questão que baseia o estudo, definida em~\ref{int:que}.
    \item \textbf{Proposições:} Como a utilização de \textit{Software Analytics} pode ajudar nas
    nas tomadas de decisões de priorização de \textit{Release Planning}.
    \item \textbf{Unidade de Análise:} Neste trabalho, um caso é definido como um repositório
        de código cuja organização é feita dentro do repositório nas Wikis, \textit{milestones}
        e \textit{issues}.
\end{itemize}

\todo[inline, backgroundcolor=yellow!20!white, bordercolor=red]{INcluir um parágrafo que faça o link de idéias entre esse capítulo e o próximo. Isso é uma técnica de escrita que situa o leitor quando da mudança de assuntos em capítulos}



